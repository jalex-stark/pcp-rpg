\documentclass[a4paper]{report}

\usepackage{amsmath,amssymb,amsthm}
\usepackage{hyperref}
\usepackage{listings}
\usepackage{xcolor}

% Lean code listings
\lstdefinelanguage{Lean}{
  morekeywords={theorem,lemma,def,example,structure,class,instance,where,by,have,show,sorry,admit},
  sensitive=true,
  morecomment=[l]{--},
  morecomment=[s]{/-}{-/},
  morestring=[b]",
}

\lstset{
  language=Lean,
  basicstyle=\small\ttfamily,
  showstringspaces=false,
  breaklines=true,
  frame=single
}

% Theorem environments
\newtheorem{theorem}{Theorem}[chapter]
\newtheorem{lemma}[theorem]{Lemma}
\newtheorem{definition}[theorem]{Definition}

% Blueprint-specific commands
\newcommand{\lean}[1]{\texttt{#1}}
\newcommand{\leanok}{\textcolor{green}{$\checkmark$}}
\newcommand{\uses}[1]{\textit{Uses: #1}}

\title{PCP Theorem Formalization Blueprint}
\author{PCP-RPG Team}
\date{\today}

\begin{document}

\maketitle

\section*{Overview}

This blueprint documents the formalization of the PCP Theorem (\textbf{NP = PCP(log n, 1)}) in Lean 4, following Irit Dinur's gap amplification proof.

\subsection*{What is the PCP Theorem?}

The Probabilistically Checkable Proof (PCP) Theorem states that every problem in NP has a proof system where:
\begin{itemize}
\item The verifier uses $O(\log n)$ random bits
\item The verifier queries only $O(1)$ bits of the proof
\item Completeness: correct statements have proofs accepted with probability 1
\item Soundness: incorrect statements are rejected with high probability
\end{itemize}

This remarkable result connects computational complexity to hardness of approximation and has profound implications for optimization problems.

\subsection*{Our Approach}

We formalize Dinur's proof via gap amplification on constraint graphs:

\begin{enumerate}
\item \textbf{Start:} NP-complete problem (SAT) encoded as constraint satisfaction
\item \textbf{Iterate:} Apply graph powering + composition to amplify the gap
\item \textbf{Result:} Constant-gap CSP, which gives PCP(log n, 1)
\end{enumerate}

\subsection*{Blueprint Structure}

The formalization is organized into 6 work packages (45 nodes total):

\begin{description}
\item[WP-A: Foundations] Core PCP definitions, languages, complexity classes
\item[WP-B: Expanders \& Preprocessing] Spectral graph theory, expander constructions
\item[WP-C: Powering] Graph powering that amplifies UNSAT gap by $\sim\sqrt{t}$
\item[WP-D: Assignment Testers] Long codes and linearity testing
\item[WP-E: Amplification Main] Combining all pieces into Dinur's main theorem
\item[WP-F: Equivalences \& Endgame] Final reduction from SAT to PCP
\end{description}

\subsection*{Status \& Progress}

\begin{itemize}
\item \textbf{Total LOC estimate:} $\sim$7,350 lines
\item \textbf{Current status:} Planning phase (all nodes in \texttt{planned} state)
\item \textbf{Repository:} \url{https://github.com/jalex-stark/pcp-rpg}
\item \textbf{Interactive graph:} Run \texttt{make graph} to visualize dependencies
\end{itemize}

\subsection*{Navigation}

Use the table of contents above or the sidebar to navigate to specific work packages. Each theorem/lemma includes:
\begin{itemize}
\item Lean signature and file path
\item Difficulty rating (1-5 stars)
\item Estimated lines of code
\item Dependencies (\texttt{uses} clauses)
\item References to Dinur's paper
\end{itemize}

\tableofcontents

% This file is auto-generated from website/data/pcp-graph.json
% by website/lib/generate_blueprint.py
% DO NOT EDIT DIRECTLY - edit the JSON and regenerate

\chapter{Introduction}

Formalization plan for NP = PCP(log n, 1) following Dinur's gap amplification proof

This blueprint presents a formalization plan for the classical PCP theorem
following Dinur's gap amplification proof via constraint graphs.

\chapter{Foundations}

PCP/Defs, Language, basic BinaryCSP + UNSAT

\textbf{Difficulty:} ★–★★

\begin{definition}[PCP Definitions]
\label{pcp-defs}
\lean{PCP.Defs}
Core PCP verifier definitions, oracle interface, PCP class predicate

\noindent Difficulty: \textbf{★★☆☆☆} $\bullet$ Est. 150 LOC $\bullet$ Work Package: Foundations

\end{definition}

\begin{definition}[Language Encodings]
\label{pcp-language}
\lean{PCP.Language}
Bitstring encodings, input representations, SAT language definition

\noindent Difficulty: \textbf{★☆☆☆☆} $\bullet$ Est. 80 LOC $\bullet$ Work Package: Foundations

\end{definition}

\begin{definition}[Constraint Graph Definitions]
\label{constraint_graph-defs}
\lean{PCP.ConstraintGraph.Defs}
Binary CSP, assignments, satisfaction, UNSAT value

\noindent Difficulty: \textbf{★★☆☆☆} $\bullet$ Est. 200 LOC $\bullet$ Work Package: Foundations

\end{definition}

\begin{definition}[UNSAT Value]
\label{constraint_graph-unsat}
\lean{PCP.ConstraintGraph.Defs}
\uses{constraint_graph-defs}
Minimum fraction of unsatisfied constraints over all assignments

\begin{lstlisting}[language=Lean]
def BinaryCSP.unsat {V α} [Fintype V] [Fintype α] (G : BinaryCSP V α) : ℚ
\end{lstlisting}

\noindent Difficulty: \textbf{★☆☆☆☆} $\bullet$ Est. 40 LOC $\bullet$ Work Package: Foundations

\textbf{Notes:} Uses finite maximization via [Fintype (V → α)]

\end{definition}

\begin{theorem}[3-Colorability is NP-hard]
\label{constraint_graph-three_color_np_hard}
\lean{PCP.ConstraintGraph.Defs}
\uses{constraint_graph-defs}
Reduction from 3-Colorability to 2-CSP satisfiability

\begin{lstlisting}[language=Lean]
theorem threeColor_to_csp : 3Color ≤p (G ↦ (UNSAT(G)=0))
\end{lstlisting}

\noindent Difficulty: \textbf{★☆☆☆☆} $\bullet$ Est. 100 LOC $\bullet$ Work Package: Foundations

\textbf{References:}
\begin{itemize}
  \item Dinur, §Proposition 1.4 (p. 3)
\end{itemize}

\end{theorem}

\chapter{Expanders \textbackslash{}& Preprocessing}

Expander graphs, spectral bounds, prep1/prep2

\textbf{Difficulty:} ★★★–★★★★

\begin{definition}[Expander Graph Definitions]
\label{expander-defs}
\lean{PCP.Expander.Defs}
Combinatorial expansion, spectral gap, explicit families

\noindent Difficulty: \textbf{★★★★☆} $\bullet$ Est. 300 LOC $\bullet$ Work Package: Expanders & Preprocessing

\textbf{Notes:} Major missing mathlib4 dependency - may need to port from Isabelle AFP

\end{definition}

\begin{theorem}[Cheeger's Inequality]
\label{expander-cheeger}
\lean{PCP.Expander.Spectral}
\uses{expander-defs}
Relates spectral gap to edge expansion

\begin{lstlisting}[language=Lean]
theorem cheeger_inequality : ∃ h, λ₁(G) ≤ d - h²/(2*d)
\end{lstlisting}

\noindent Difficulty: \textbf{★★★★★} $\bullet$ Est. 200 LOC $\bullet$ Work Package: Expanders & Preprocessing

\textbf{Notes:} Core spectral graph theory - needs Rayleigh quotient formalization

\textbf{References:}
\begin{itemize}
  \item Dinur, §Theorem 2.3 (p. 8)
\end{itemize}

\end{theorem}

\begin{definition}[Preprocess (Regularization + Expanderization)]
\label{constraint_graph-preprocess}
\lean{PCP.ConstraintGraph.Preprocess}
\uses{expander-cheeger, constraint_graph-defs}
Transform CSP into regular, expanding graph via vertex clouds and edge addition

\begin{lstlisting}[language=Lean]
def preprocess {V α} [Fintype V] [Fintype α] (G : BinaryCSP V α) : Σ (V' : Type*) [Fintype V'], BinaryCSP V' α
\end{lstlisting}

\noindent Difficulty: \textbf{★★★★☆} $\bullet$ Est. 250 LOC $\bullet$ Work Package: Expanders & Preprocessing

\textbf{References:}
\begin{itemize}
  \item Dinur, §Def. 4.1, Lemma 4.1-4.2 (pp. 12-14)
\end{itemize}

\end{definition}

\begin{lemma}[Preprocessing Preserves UNSAT]
\label{constraint_graph-preprocess_spec}
\lean{PCP.ConstraintGraph.Preprocess}
\uses{constraint_graph-preprocess}
Preprocessing maintains UNSAT up to constant factor while achieving regularity and expansion

\begin{lstlisting}[language=Lean]
lemma preprocess_spec : ∃ λ d β₁, β₁*UNSAT(G) ≤ UNSAT(preprocess G) ≤ UNSAT(G) ∧ regular ∧ λ ≤ const
\end{lstlisting}

\noindent Difficulty: \textbf{★★★★☆} $\bullet$ Est. 300 LOC $\bullet$ Work Package: Expanders & Preprocessing

\textbf{References:}
\begin{itemize}
  \item Dinur, §Lemma 1.7 (pp. 5, 10-12)
\end{itemize}

\end{lemma}

\chapter{Powering}

Graph powering + soundness theorem (hardest piece)

\textbf{Difficulty:} ★★★★★

\begin{definition}[Graph Powering]
\label{constraint_graph-power}
\lean{PCP.ConstraintGraph.Powering}
\uses{constraint_graph-defs}
G\textbackslash{}^{}t with walk-based constraints; alphabet grows to tuples

\begin{lstlisting}[language=Lean]
def power {V α} [Fintype V] [Fintype α] (t : ℕ) (G : BinaryCSP V α) : BinaryCSP V (Fin (Gdegree G) → α)
\end{lstlisting}

\noindent Difficulty: \textbf{★★★☆☆} $\bullet$ Est. 200 LOC $\bullet$ Work Package: Powering

\textbf{References:}
\begin{itemize}
  \item Dinur, §§1.2 (pp. 4-5)
\end{itemize}

\end{definition}

\begin{theorem}[Powering Amplifies UNSAT]
\label{constraint_graph-powering_soundness}
\lean{PCP.ConstraintGraph.Powering}
\uses{constraint_graph-power, expander-cheeger}
Core gap amplification: powering increases UNSAT by \textbackslash{}textasciitilde{}√t

\begin{lstlisting}[language=Lean]
theorem powering_soundness : ∃ β₂>0, UNSAT(G^t) ≥ β₂ * √t * min (UNSAT(G)) (1/t)
\end{lstlisting}

\noindent Difficulty: \textbf{★★★★★} $\bullet$ Est. 400 LOC $\bullet$ Work Package: Powering

\textbf{Notes:} HARDEST LEMMA - requires random walk mixing bounds and spectral inequalities

\textbf{References:}
\begin{itemize}
  \item Dinur, §Lemma 1.6 (pp. 4-5)
\end{itemize}

\end{theorem}

\chapter{Assignment Testers}

Defs + Existence (Long Code tester)

\textbf{Difficulty:} ★★★–★★★★

\begin{definition}[Assignment Tester Interface]
\label{assignment_tester-defs}
\lean{PCP.AssignmentTester.Defs}
\uses{constraint_graph-defs}
Assignment tester structure with soundness and completeness properties

\noindent Difficulty: \textbf{★★★☆☆} $\bullet$ Est. 150 LOC $\bullet$ Work Package: Assignment Testers

\end{definition}

\begin{theorem}[Composition Reduces Alphabet]
\label{assignment_tester-composition}
\lean{PCP.AssignmentTester.Defs}
\uses{assignment_tester-defs}
Composing CSP with assignment tester preserves UNSAT while reducing alphabet

\begin{lstlisting}[language=Lean]
theorem composition_lemma : ∃ β₃>0 c, UNSAT(G ∘ P) ≥ β₃ * UNSAT(G) ∧ size ≤ c * size(G)
\end{lstlisting}

\noindent Difficulty: \textbf{★★★☆☆} $\bullet$ Est. 250 LOC $\bullet$ Work Package: Assignment Testers

\textbf{References:}
\begin{itemize}
  \item Dinur, §Lemma 1.8, Def. 5.1 (pp. 16-17)
\end{itemize}

\end{theorem}

\begin{theorem}[Constant-Alphabet Tester Exists]
\label{assignment_tester-existence}
\lean{PCP.AssignmentTester.Existence}
\uses{assignment_tester-defs}
Explicit construction via Long Code

\begin{lstlisting}[language=Lean]
theorem tester_exists : ∃ P : AssignmentTester, Fintype.card P.Σ0 = O(1) ∧ P.ε ≥ ε0
\end{lstlisting}

\noindent Difficulty: \textbf{★★★★☆} $\bullet$ Est. 300 LOC $\bullet$ Work Package: Assignment Testers

\textbf{Notes:} Decouplable module - can be treated as black box initially

\textbf{References:}
\begin{itemize}
  \item Dinur, §Theorem 5.1, §7 (pp. 16, 23+)
\end{itemize}

\end{theorem}

\chapter{Amplification Main}

Glue C+B+D to prove dinur\textbackslash{}_main

\textbf{Difficulty:} ★★★

\begin{theorem}[Dinur Main Theorem]
\label{amplification-main}
\lean{PCP.Amplification.Main}
\uses{constraint_graph-preprocess_spec, constraint_graph-powering_soundness, assignment_tester-composition, assignment_tester-existence}
Gap-doubling with linear size growth - combines preprocessing, powering, and composition

\begin{lstlisting}[language=Lean]
theorem dinur_main : ∃ Σ0, ∀ Σ, ∃ C α∈(0,1), ∀ G, ∃ G', size(G') ≤ C*size(G) ∧ UNSAT(G') ≥ min(2*UNSAT(G), α)
\end{lstlisting}

\noindent Difficulty: \textbf{★★★☆☆} $\bullet$ Est. 200 LOC $\bullet$ Work Package: Amplification Main

\textbf{References:}
\begin{itemize}
  \item Dinur, §Theorem 1.5 (pp. 10-11)
\end{itemize}

\end{theorem}

\chapter{Equivalences \textbackslash{}& Endgame}

Gap-CSP ⇔ PCP, trivial inclusion, final statement

\textbf{Difficulty:} ★★

\begin{theorem}[PCP(log n, 1) ⊆ NP]
\label{pcp-subset_np}
\lean{PCP.Defs}
\uses{pcp-defs}
Trivial inclusion: enumerate all random strings and include proof bits in NP certificate

\begin{lstlisting}[language=Lean]
theorem PCP_subset_NP : ∀ L, PCP (λ n, O(log n)) (λ _, O(1)) L → L ∈ NP
\end{lstlisting}

\noindent Difficulty: \textbf{★☆☆☆☆} $\bullet$ Est. 50 LOC $\bullet$ Work Package: Equivalences & Endgame

\textbf{References:}
\begin{itemize}
  \item Arora-Barak, §Ch. 11 (Remark 11.6(3))
\end{itemize}

\end{theorem}

\begin{theorem}[Gap-CSP ⇔ PCP(log n, O(1))]
\label{equivalences-gap_pcp}
\lean{PCP.Equivalences}
\uses{pcp-defs, constraint_graph-defs}
Standard equivalence between PCP and Gap-CSP formulations

\begin{lstlisting}[language=Lean]
theorem PCP_gapCSP_equiv : (NP ⊆ PCP(log, O(1))) ↔ (Gap-2CSP with constant gap is NP-hard)
\end{lstlisting}

\noindent Difficulty: \textbf{★★☆☆☆} $\bullet$ Est. 150 LOC $\bullet$ Work Package: Equivalences & Endgame

\textbf{References:}
\begin{itemize}
  \item Dinur, §Lemma 1.3 (pp. 2-3)
\end{itemize}

\end{theorem}

\begin{theorem}[NP-hard Constant-Gap 2-CSP]
\label{endgame-gap_csp_hard}
\lean{PCP.Endgame}
\uses{amplification-main, constraint_graph-three_color_np_hard}
Iterate Dinur main theorem O(log n) times to boost gap to constant

\begin{lstlisting}[language=Lean]
theorem gap2csp_hard : ∃ Σ0, NP-hard ({G | UNSAT(G)=0} vs UNSAT(G)≥α)
\end{lstlisting}

\noindent Difficulty: \textbf{★★☆☆☆} $\bullet$ Est. 100 LOC $\bullet$ Work Package: Equivalences & Endgame

\textbf{References:}
\begin{itemize}
  \item Dinur, §Theorem 1.2 (pp. 11-12)
\end{itemize}

\end{theorem}

\begin{theorem}[NP = PCP(log n, 1)]
\label{endgame-pcp_theorem}
\lean{PCP.Endgame}
\uses{equivalences-gap_pcp, endgame-gap_csp_hard, pcp-subset_np}
Final statement combining all pieces

\begin{lstlisting}[language=Lean]
theorem PCP_theorem : NP = { L | PCP (λ n, O(log n)) (λ _, O(1)) L }
\end{lstlisting}

\noindent Difficulty: \textbf{★★☆☆☆} $\bullet$ Est. 50 LOC $\bullet$ Work Package: Equivalences & Endgame

\textbf{References:}
\begin{itemize}
  \item Arora-Barak, §Theorem 11.5
\end{itemize}

\end{theorem}


\end{document}